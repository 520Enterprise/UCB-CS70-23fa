\documentclass[11pt]{article}
\usepackage{header}
\def\title{HW 03}

\begin{document}
\maketitle
\fontsize{12}{15}\selectfont

\begin{center}
    Due: Saturday, 9/16, 4:00 PM \\
    Grace period until Saturday, 9/16, 6:00 PM \\
\end{center}

\section*{Sundry}
Before you start writing your final homework submission, state briefly how you worked on it.  Who else did you work with?  List names and email addresses.  (In case of homework party, you can just describe the group.)
\begin{solution}
    by myself
\end{solution}

\vspace{15pt}

\Question{Build-Up Error?}

\notelinks{\href{https://www.eecs70.org/assets/pdf/notes/n5.pdf}{Note 5}}
What is wrong with the following "proof"? In addition to finding a counterexample, you should explain what is fundamentally wrong with this approach, and why it demonstrates the danger of build-up error.

\textbf{False Claim:}~If every vertex in an undirected graph has degree at least 1, then the graph is connected.

\begin{proof}[Proof?]
  We use induction on the number of vertices $n \ge 1$.

\emph{Base case:} There is only one graph with a single vertex and it has degree 0. Therefore, the base case is vacuously true, since the if-part is false.

\emph{Inductive hypothesis:} Assume the claim is true for some $n \ge 1$.

\emph{Inductive step:} We prove the claim is also true for $n+1$. Consider an undirected graph on $n$ vertices in which every vertex has degree at least 1. By the inductive hypothesis, this graph is connected. Now add one more vertex $x$ to obtain a graph on $(n + 1)$ vertices, as shown below.
\begin{center}
  \begin{tikzpicture}[node distance=0pt]
    \node at (1.5, 3.5) {$n$-vertex graph};
    \draw[dotted] plot[smooth cycle] coordinates {(0, 0) (0.5, 3) (2.5, 2.5) (2, 0)};
    \node[circ] (z) at (1.5, 2.5) {};
    \node[right=of z] {$z$};
    \node[circ] (y) at (1, 0.5) {};
    \node[below=of y] {$y$};
    \draw[dashed] plot[smooth,tension=0.9] coordinates {(z) (1, 2) (1.5, 1) (y)};
    \node[circ] (x) at (-1, 1) {};
    \node[above=of x] {$x$};
    \draw (x) -- (y);
  \end{tikzpicture}
\end{center}
All that remains is to check that there is a path from $x$ to every other vertex $z$. Since $x$
has degree at least 1, there is an edge from $x$ to some other vertex; call it $y$. Thus, we
can obtain a path from $x$ to $z$ by adjoining the edge $\{x,y\}$ to the path from $y$ to $z$. This
proves the claim for $n+1$. 
\end{proof}

\begin{solution}
    \\
    \textbf{Counterexample:} Consider the following graph with 4 vertices, each with degree at least 1. The graph is not connected.
    \begin{center}
        \begin{tikzpicture}[node distance=0pt]
        \node[circ] (x) at (-1, 1) {};
        \node[above=of x] {$x$};
        \node[circ] (y) at (1, 0.5) {};
        \node[below=of y] {$y$};
        \node[circ] (z) at (1.5, 2.5) {};
        \node[right=of z] {$z$};
        \node[circ] (w) at (2.5, 2.5) {};
        \node[right=of w] {$w$};
        \draw (x) -- (y); 
        \draw (z) -- (w);
        \end{tikzpicture}
    \end{center}
    \textbf{Explanation:} We cannot add a vertex because we cannot achieve every possible graph with $n+1$ vertices by adding a vertex to a graph with $n$ vertices. For example, the graph with 4 vertices above cannot be achieved by adding a vertex to a valid graph with 3 vertices below.
\end{solution}

\Question{Proofs in Graphs} 

\notelinks*{\href{https://www.eecs70.org/assets/pdf/notes/n5.pdf}{Note 5}}
\begin{Parts}

\Part On the axis from San Francisco traffic habits to Los Angeles traffic habits, Old California is more towards San Francisco: that is, civilized. In Old California, all roads were one way streets. Suppose Old California had 
$n$ cities ($n \geq 2$) such that for every pair of cities $X$ and $Y$,
either $X$ had a road to $Y$ or $Y$ had a road to $X$.

Prove that there existed a city which was reachable from every other city by traveling through at most 2 roads. 

[\textit{Hint:} Induction]

\Part Consider a connected graph $G$ with $n$ vertices which has exactly $2m$ vertices of
odd degree, where $m > 0$. Prove that there are $m$ walks that \emph{together} 
cover all the edges of $G$ (i.e., each edge of $G$ occurs in exactly one of the $m$ walks, 
and each of the walks should not contain any particular edge more than once).

[\emph{Hint:} In lecture, we have shown that a connected undirected graph has an Eulerian tour if and only if every vertex has even degree. This fact may be useful in the proof.]

\Part Prove that any graph $G$ is bipartite if and only if it has no tours of odd length.

[\emph{Hint:} In one of the directions, consider the lengths of paths starting from a given vertex.]
\end{Parts}

\begin{solution}
    \begin{Parts}
        \Part
        \begin{proof}
            We use induction on $n$.
            
            \emph{Base case:} $n = 2$. There are two cities, $X$ and $Y$. Either $X$ has a road to $Y$ or $Y$ has a road to $X$. In either case, there is a city which is reachable from every other city by traveling through at most 2 roads.
            
            \emph{Inductive hypothesis:} Assume the claim is true for some $n \ge 2$.
            
            \emph{Inductive step:} 
            We choose a vertex $Z$ that have a road to every other vertex and delete it. By the inductive hypothesis, there is a vertex $X$ which is reachable from every other city by traveling through at most 2 roads. If $Z$ has a road to $X$, then $X$ is reachable from every other city by traveling through at most 2 roads. Otherwise $X$ has a road to $Z$. There must be a city $Y$ which is reachable from $Z$ by one road, if $Y$ has a road to $X$, then $X$ is reachable from $Z$. Otherwise $X$ has a road to $Y$. Then there must be $Y->A->X$ by the inductive hypothesis. And if $Z$ has a road to $A$, then $X$ is reachable from $Z$. Otherwise for all $Y$, we can find $Y->A->Z$, $Z$ is the vertex we want to find.
        \end{proof}
        
        \Part
        \begin{proof}
            Add $m$ edges between pairs of odd degree vertices. Then we get a graph with all even degree vertices. 
            By the hint, the graph has an Eulerian tour. Then we can get $m$ walks that together cover all the edges we added. 
            Then we can get $m$ walks that together cover all the edges of $G$.
        \end{proof}

        \Part
        \begin{proof}
            If there is a tour of odd length, then there is a cycle of odd length. We can color them one by one and can achieve contradiction. So any graph $G$ is bipartite if it has no tours of odd length.
            Then we prove the other direction. 
            We choode a vertex $x$ then devide the graph into two parts, $A$ and $B$. $A$ contains all vertices that can be reached from $x$ by even length paths. $B$ contains all vertices that can be reached from $x$ by odd length paths. 
            They have different colors. 
            If $G$ is not bipartite, there must be an edge between 2 vertices in $A$ or 2 vertices in $B$.
            Without loss of generality, we assume there is an edge between $a_1$ and $a_2$ in $A$.
            Then the tour $x->a_1->a_2->x$ is odd length. So $G$ has a tour of odd length.
        \end{proof}
    \end{Parts}
\end{solution}

\Question{Binary Trees}

\notelinks{\href{https://www.eecs70.org/assets/pdf/notes/n5.pdf}{Note 5}}
You may have seen the recursive definition of binary trees from previous classes. Here, we define binary trees in graph theoretic terms as follows (\textbf{Note:} here we will modify the definition of leaves slightly for consistency). 
\begin{itemize}
	\item A binary tree of height $> 0$ is a tree where exactly one vertex, called the \textbf{root}, has degree $2$, and all other vertices have degrees $1$ or $3$. Each vertex of degree $1$ is called a \textbf{leaf}. The \textbf{height} $h$ is defined as the maximum length of the path between the root and any leaf.
	\item A binary tree of height $0$ is the graph with a single vertex. The vertex is both a leaf and a root.
\end{itemize}
\begin{Parts}
	\Part Let $T$ be a binary tree of height $> 0$, and let $h(T)$ denote its height. Let $r$ be the root in $T$ and $u$ and $v$ be its neighbors. Show that removing $r$ from $T$ will result in two binary trees, $L,R$ with roots $u$ and $v$ respectively. Also, show that $h(T) = \max(h(L),h(R)) + 1$.
		
	\Part Using the graph theoretic definition of binary trees, prove that the number of vertices in a binary tree of height $h$ is at most $2^{h+1}-1$.
	
	\Part Prove that all binary trees with $n$ leaves have $2n-1$ vertices.
\end{Parts}

\begin{solution}
\begin{Parts}
    \Part 
    \begin{proof}
        By the definition, after we remove $r$, $u$ and $v$ have 0 or 2 neighbors. Other vertices have 1 or 3 neighbors. So $L$ and $R$ are binary trees. And $h(T) = \max(h(L),h(R)) + 1$ because $h(T)$ is the maximum length of the path between the root and any leaf, and after we remove the root, the maximum length of the path between the root and any leaf is the maximum length of the path between the root of $L$ and any leaf of $L$ or the maximum length of the path between the root of $R$ and any leaf of $R$.
    \end{proof}
    \Part 
    \begin{proof}
        We use induction on $h$.
        
        \emph{Base case:} $h = 0$. There is only one vertex in the binary tree. So the number of vertices is $2^{0+1}-1 = 1$.
        
        \emph{Inductive hypothesis:} Assume the claim is true for some $h \ge 0$.
        
        \emph{Inductive step:} We prove the claim is also true for $h+1$. Consider a binary tree of height $h+1$. By the inductive hypothesis, the number of vertices in the left binary tree is at most $2^{h+1}-1$ and the number of vertices in the right binary tree is at most $2^{h+1}-1$. So the number of vertices in the binary tree is at most $2^{h+1}-1 + 2^{h+1}-1 + 1 = 2^{(h+1)+1}-1$. This proves the claim for $h+1$.
    \end{proof}
    \Part
    \begin{proof}
        We use induction on $n$.
        
        \emph{Base case:} $n = 1$. There is only one leaf in the binary tree. So the number of vertices is $2*1-1 = 1$.
        
        \emph{Inductive hypothesis:} Assume the claim is true for some $n \ge 1$.
        
        \emph{Inductive step:} We prove the claim is also true for $n+1$. Consider a binary tree with $n+1$ leaves. To remove a leaf we have to remove 2 original leaves (then theirs father will become a leaf). 
        By the inductive hypothesis, the number of vertices in the binary tree is $2n-1$. So the number of vertices in the binary tree with $n+1$ leaves is $2n-1 + 2 = 2(n+1)-1$. This proves the claim for $n+1$.
    \end{proof}
\end{Parts}
\end{solution}

\Question{Touring Hypercube} 

\notelinks{\href{https://www.eecs70.org/assets/pdf/notes/n5.pdf}{Note 5}}
In the lecture, you have seen that if $G$ is a hypercube of dimension $n$, then
\begin{itemize}
    \item The vertices of $G$ are the binary strings of length $n$.
    \item $u$ and $v$ are connected by an edge if they differ in exactly one bit location.
\end{itemize}

A \emph{Hamiltonian tour} of a graph is a sequence of vertices
$v_0, v_1, \ldots, v_k$ such that:
\begin{itemize}
    \item Each vertex appears exactly once in the sequence.
    \item Each pair of consecutive vertices is connected by an edge.
    \item $v_0$ and $v_k$ are connected by an edge.
\end{itemize}

\begin{Parts}

    \Part Show that a hypercube has an Eulerian tour if and only if $n$ is even.
    

    \Part Show that every hypercube has a Hamiltonian tour. 

    

\end{Parts}

\begin{solution}
\begin{Parts}
    \Part
    \begin{proof}
        Every vertex in a hypercube has degree $n$. So the hypercube has an Eulerian tour if and only if $n$ is even.
    \end{proof}
    \Part
    \begin{proof}
        We use induction on $n$.
        
        \emph{Base case:} $n = 1$. There are two vertices in the hypercube, $0$ and $1$. There is a Hamiltonian tour, $0->1$.
        
        \emph{Inductive hypothesis:} Assume the claim is true for some $n \ge 1$.
        
        \emph{Inductive step:} We prove the claim is also true for $n+1$. 
        Consider a hypercube of dimension $n+1$. We devide this hypercube into two hypercubes of dimension $n$. 
        By the inductive hypothesis, the left hypercube has a Hamiltonian tour, $v_0, v_1, \ldots, v_k$, 
        and the right hypercube has a Hamiltonian tour, $v^{'}_0, v^{'}_1, \ldots, v^{'}_k$.
        Then we can get a Hamiltonian tour of the hypercube of dimension $n+1$, $v_0, v_1, \ldots, v_k, v^{'}_k, v^{'}_{k-1}, \ldots, v^{'}_0$.
    \end{proof}
\end{Parts}
\end{solution}





\Question{Edge Colorings}

\notelinks{\href{https://www.eecs70.org/assets/pdf/notes/n5.pdf}{Note 5}}
An edge coloring of a graph is an assignment of colors to edges in a graph where any two edges incident to the same vertex have different colors. An example is shown on the left.

\begin{center}
    \begin{tikzpicture}
        \clip (-1, -1) rectangle (8, 2.1);

        \node[circ] (n1) at (0, 0) {};
        \node[circ] (n2) at (1, {sqrt(3)}) {};
        \node[circ] (n3) at (2, 0) {};
        \draw (n1) -- node[above left] {color 1} (n2)
        -- node[above right] {color 2} (n3)
        -- node[below] {color 3} (n1);

        \node[circ] (m1) at (5, 0) {};
        \node[circ] (m2) at (5, 2) {};
        \node[circ] (m3) at (7, 2) {};
        \node[circ] (m4) at (7, 0) {};
        \draw (m1) -- (m2) -- (m3) -- (m4) -- (m1);
        \draw (m2) -- (m4);
        \draw (m1) edge[out=-60, in=-30, looseness=2.5] (m3);
    \end{tikzpicture}
\end{center}

\begin{Parts}
\Part Show that the 4 vertex complete graph above can be 3 edge colored. (Use the numbers $1,2,3$ for colors. A figure is shown on the right.)





\Part Prove that any graph with maximum degree $d \geq 1$ can be edge colored with $2d-1$ colors. 



\Part Show that a tree can be edge colored with $d$ colors where $d$ is the maximum degree of any vertex.

\end{Parts}

\begin{solution}
\begin{Parts}
    \Part
    \begin{center}
        \begin{tikzpicture}
            \clip (-1, -1) rectangle (8, 2.1);
    
            \node[circ] (m1) at (5, 0) {};
            \node[circ] (m2) at (5, 2) {};
            \node[circ] (m3) at (7, 2) {};
            \node[circ] (m4) at (7, 0) {};
            \draw (m1) -- (m2) -- (m3) -- (m4) -- (m1);
            \draw (m2) -- (m4);
            \draw (m1) edge[out=-60, in=-30, looseness=2.5] (m3);
            % Label the edges with colors
            \draw (m1) -- node[left] {1} (m2);
            \draw (m2) -- node[below] {2} (m3);
            \draw (m3) -- node[right] {1} (m4);
            \draw (m4) -- node[below] {2} (m1);
            \draw (m2) -- node[right] {3} (m4);
            \draw (m1) edge[out=-60, in=-30, looseness=2.5] node[below] {3} (m3);
        \end{tikzpicture}
    \end{center}
    \Part
    \begin{proof}
        No idea
    \end{proof}
    \Part
    \begin{proof}
        We can color them through a BFS process from the vertex which have maximum degree ($d$ colors).
        When we access a vertex, we color the edge between it and its children with a new color from each other.
    \end{proof}
\end{Parts}
\end{solution}





\Question{Planarity and Graph Complements}

\notelinks{\href{https://www.eecs70.org/assets/pdf/notes/n5.pdf}{Note 5}}
Let $G = (V, E)$ be an undirected graph.  We define the complement of $G$ as $\overline{G} = (V, \overline{E})$ where $\overline{E} = \{(i,j) \mid i,j \in V, i \neq j\} - E$; that is, $\overline{G}$ has the same set of vertices as $G$, but an edge $e$ exists is $\overline{G}$ if and only if it does not exist in $G$.

\begin{Parts}

\Part Suppose $G$ has $v$ vertices and $e$ edges.  How many edges does $\overline{G}$ have?


\Part Prove that for any graph with at least 15 vertices, $G$ being planar implies that $\overline{G}$ is non-planar.


\Part Now consider the converse of the previous part, i.e., for any graph $G$ with at least 15 vertices, if $\overline{G}$ is non-planar, then $G$ is planar. Construct a counterexample to show that the converse does not hold.

\textit{Hint: Recall that if a graph contains a copy of $K_5$, then it is non-planar. Can this fact be used to construct a counterexample?}


\end{Parts}

\begin{solution}
\begin{Parts}
    \Part
        Since $\overline{E} = \{(i,j) \mid i,j \in V, i \neq j\} - E$, $\overline{G}$ has $v$ vertices and $\frac{v(v-1)}{2} - e$ edges.
    \Part
    \begin{proof}
        Since $G$ is planar, $e \le 3v-6$. Since $\overline{G}$ has $v$ vertices and $\frac{v(v-1)}{2} - e$ edges, $\frac{v(v-1)}{2} - e - (3v-6) \ge \frac{v(v-1)}{2} -2(3v-6) = \frac{v(v-1)}{2} -6v + 12 = \frac{v^2-13v+24}{2} = \frac{(v-1)(v-12)}{2} + 6 \ge 0$. So $\overline{G}$ is non-planar.
    \end{proof}
    \Part
    \textbf{Counterexample:}
    Let $\overline{G}$ be $K_5$, then $G$ is non-planar. But $G$ is obviously also not planar.(contain $K_{10}$ !!!)
\end{Parts}
\end{solution}


\end{document}
